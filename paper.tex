\documentclass[conference]{IEEEtran}

% correct bad hyphenation here
\hyphenation{op-tical net-works semi-conduc-tor}


\usepackage{float}
\usepackage[utf8]{inputenc}
\usepackage{url}
\usepackage{array}
\usepackage{cite}

% Centered columns
\newcolumntype{V}{>{\bf\centering\arraybackslash} m{.2\linewidth} }

% Default fixed font does not support bold face
\DeclareFixedFont{\ttb}{T1}{txtt}{bx}{n}{12} % for bold
\DeclareFixedFont{\ttm}{T1}{txtt}{m}{n}{12}  % for normal

% Custom colors
\usepackage{color}
\definecolor{deepblue}{rgb}{0,0,0.5}
\definecolor{deepred}{rgb}{0.6,0,0}
\definecolor{deepgreen}{rgb}{0,0.5,0}

\usepackage{listings}

% Python style for highlighting
\newcommand\pythonstyle{\lstset{
language=Python,
basicstyle=\ttm,
otherkeywords={self},
keywordstyle=\ttb\color{deepblue},
emphstyle=\ttb\color{deepred},
stringstyle=\color{deepgreen},
frame=tb,
showstringspaces=false
}}


% Python environment
\lstnewenvironment{python}[1][]
{
\pythonstyle
\lstset{#1}
}
{}


\begin{document}

\title{fl0w 2.0 - a development environment for the\\ KIPR Wallaby}
\author{\IEEEauthorblockN{Philip Trauner}
\IEEEauthorblockA{School of Computer Engineering\\
HTBLuVA Wiener Neustadt\\
Email: philip.trauner@arztpraxis.io}
\and
\IEEEauthorblockN{Christoph Heiss}
\IEEEauthorblockA{School of Computer Engineering\\
HTBLuVA Wiener Neustadt\\
Email: me@christoph-heiss.me}}

\maketitle


\begin{abstract}
This publication introduces fl0w, an alternative\\ development environment for the KIPR Wallaby.\\ The aim of fl0w is to improve the robotics program development experience. It focuses on the components that make up fl0w, namely a file synchronization protocol that maintains a consistent state across all connected controllers, a route based and data type preserving network protocol with peer-to-peer piping capabilities, a discovery protocol that connects all controllers together automatically, a Sublime Text 3\cite{Sublime Text 3:Sublime HQ} plugin which enables in-line sensor readouts, program execution, program editing, and keyboard shortcuts, and a browser-based management front-end to manage the controller fleet.\\
\end{abstract}

\begin{IEEEkeywords}
file synchronization, networking, LAN discovery, development environment
\end{IEEEkeywords}



\section{Introduction}
fl0w was developed out of a need for a fast, reliable and wireless workflow solution that can compete with the currently available offerings like Harrogate\cite{Harrogate:KIPR}. Its goals are to transform the connected controllers to a redundant fleet of logical units that share the same binaries and source code, real-time in-line sensor readouts, and a management layer inside the text editor as well as a browser-based interface. 

\section{Implementation}
All fl0w components emphasize shared code and are therefor written in Python 3.3.6\cite{Python 3.3.6:Python Foundation} to remain compatible with the Sublime Text 3\cite{Sublime Text 3:Sublime HQ} plugin environment. Python was chosen as the implementation language for fl0w because of its rich standard library and its lightweight virtual machine that can run on a Wallaby controller with acceptable speed. A pre-compiled version of Python 3.3.6 \cite{Python 3.3.6:Python Foundation} is bundled with the fl0w installer because it is not present on the Wallaby by default. All fl0w components are designed as libraries to allow for integration into other projects.

\section{Components}
\subsection{undergr0und}
undergr0und\cite{undergr0und:Philip Trauner} is an asynchronous route based and data type preserving network protocol with peer-to-peer piping capabilities. 
It automatically converts data into a transferable format and prepends binary headers to allow for reconstruction on the other end. 

\begin{figure}[H]
\centering
	\begin{tabular}{*{3}{V}}
		Data type & Route ID & Data \\ \hline
		1 & 2 & * \\
	\end{tabular}
	\caption{The binary encoding of undergr0und (in bytes)}
\label{fig:undergr0und_header}
\end{figure}


The Python\cite{Python 3.3.6:Python Foundation} version is built on top of a fork of ws4py\cite{ws4py:Philip Trauner} and the JavaScript version utilizes regular WebSockets\cite{The WebSocket Protocol:A. Melnikov} (in arraybuffer mode). It was created to mitigate the problems fl0ws explorative development style created. Instead of one monolithic network protocol, undergr0und consists of many small sub protocols. This was done because all monolithic designs became too inconsistent. undergr0und\cite{undergr0und:Philip Trauner} manages itself through its own concepts. 

\subsubsection{Exchange table}
To reduce the required additional bandwidth per message that is introduced with variable character count routes a numbered route lookup table is generated on startup by clients and the server. Before any communication takes place these lookup tables are exchanged. Route ID 0 is reserved by undergr0und\cite{undergr0und:Philip Trauner} for self management purposes. 

\begin{figure}[H]
\centering
\begin{python}
>>> routes = {"echo": Echo(), 
...    "help": Help()}
>>> create_exchange_map(routes)
{0: "meta", 1: "echo", 2: "help"}
\end{python}
\caption{Creation of exchange maps}
\end{figure}

\subsubsection{Data type preservation}
The original type of the data segment is present inside the header $($see Figure ~\ref{fig:undergr0und_header}$)$ to accurately reconstruct sent data on the other end. JSON\cite{JSON:T. Bray Ed.} is used to transfer lists and dictionaries, and regular ASCII encoded strings are utilized to transfer integers, floats, null/none types and strings.

\subsubsection{Route}
A client$\,\to\,$server / server$\,\to\,$client construct. 

\subsubsection{Pipe}
A client$\,\to\,$server$\,\to\,$client construct. Clients can be targeted with unique IDs that are generated randomly for all connected peers by the server. There is no predefined mechanism in the network protocol that exposed these peer IDs, the application using undergr0und\cite{undergr0und:Philip Trauner} has to provide a way of making them available.
Pipe messages are packaged inside regular route messages with additional headers in the data segment. An optional server route called {\color{deepgreen}"pipe"} unpacks the regular and the extended headers and forwards the message to the targeted peer. 


\section{Conclusion}
The conclusion goes here.


% section* for acknowledgment
\section*{Acknowledgment}
The authors would like to thank the robotics team robot0nfire: Nico Kratky, Nico Leidenfrost,\\ Sebastian Schaffler, Christine Zeh, Sascha Zemann;\\ Dr. Michael Stifter for making the existence of our team possible; Daniel Maximilian Swoboda for answering all paper related questions; the KIPR development team without whom the Wallaby Controller would not exist.

\begin{thebibliography}{1}

\bibitem{Sublime Text 3:Sublime HQ}
Sublime HQ, \emph{Sublime Text 3}, \url{https://www.sublimetext.com/3},\\ product page,
accessed February 9th 2017

\bibitem{Harrogate:KIPR}
KIPR, \emph{Harrogate}, \url{https://github.com/kipr/harrogate},\\ source code,
accessed February 9th 2017

\bibitem{Python 3.3.6:Python Foundation}
Python Foundation, \emph{Python 3.3.6}, \url{https://www.python.org/downloads/release/python-336/}, source code, accessed February 9th 2017

\bibitem{undergr0und:Philip Trauner}
Philip Trauner, \emph{undergr0und}, \url{https://github.com/robot0nfire/fl0w/wiki/undergr0und}, implementation notes, accessed February 9th 2017

\bibitem{ws4py:Philip Trauner}
Philip Trauner, \emph{ws4py}, \url{https://github.com/robot0nfire/ws4py}, source code, accessed February 9th 2017

\bibitem{The WebSocket Protocol:A. Melnikov}
A. Melnikov, \emph{The WebSocket Protocol}, \url{https://tools.ietf.org/html/rfc6455}, request for comment, accessed February 9th 2017

\bibitem{JSON:T. Bray Ed.}
T. Bray, Ed., \emph{JSON}, \url{https://tools.ietf.org/html/rfc7159}, request for comment, accessed February 9th 2017



\end{thebibliography}

\end{document}